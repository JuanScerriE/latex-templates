\documentclass{uom-coursework}

\usepackage{blindtext}

\begin{document}

\title{Data Structures and\\Algorithms 2}
\tagline{Coursework}
\author{Juan Scerri}
\authorid{1234567A}
\courseworkname{Some Degree}
\doctype{coursework}
\courseworkdate{\monthyeardate\today}
\subjectcode{ICS2210}

\frontmatter
\maketitle

\tableofcontents*

\clearpage

\pagestyle{umpage}
\mainmatter
\chapter{Introduction}

Noti għal-qabel l-eżmai

\texttt{ABCDE... Is this Courier?}

To find Fibonacci sums very quickly, the two results listed
below can be used.



\url{https://en.wikipedia.org/wiki/Coffee}

\blinddocument

\begin{align}
  F_n = \round{\frac{\phi^n}{\sqrt{5}}}, \quad \phi = \frac{1 +
  \sqrt{5}}{2}. \label{fib_num}
\end{align}

\begin{align}
  \sum_{i=1}^{n} F_i = F_{n+2} - 1. \label{fib_sum}
\end{align}

\coffeestainC{0.3}{0.85}{-25}{1cm}{0cm}

The original equation for the Fibonacci numbers is a recurrence
relation defined as

\begin{align}
  F_n = F_{n-1} + F_{n-2}, \qquad F_1 = F_2 = 1. \label{fib_def}
\end{align}

However, there exists a closed-form expression called Binet's
Formula which is the following

\begin{align*}
  F_n = \frac{\phi^n - (-\phi)^n}{\sqrt{5}}.
\end{align*}

One notices that,

\begin{align*}
  \forall \, n \in \mathbb{N} : \abs{\frac{(-\phi)^n}{\sqrt{5}}}
  < \frac{1}{2},
\end{align*}

which implies

\begin{align*}
  \forall \, n \in \mathbb{N} : \abs{F_n -
  \frac{\phi^n}{\sqrt{5}}} < \frac{1}{2}.
\end{align*}

Visually, this can be represented as follows.

\begin{center}
  \begin{tikzpicture}
    \draw [thick] (-4, 0) -- (4, 0);
    \draw [thick] (0, -0.2) -- (0, 0.2);

    % Brackets
    \node at (-1.5, 0) {$($};
    \node at (1.5, 0) {$)$};

    % Bottom labels
    \node [below] at (0, -0.25) {$F_n$};
    \node [below] at (-1.5, -0.25) {$F_n - \frac{1}{2}$};
    \node [below] at (1.5, -0.25) {$F_n + \frac{1}{2}$};

    % Indicator of our value
    \draw [thick, ->] (-1, 0.5) -- (-1, 0);

    % Our value
    \node [above] at (-1, 0.5) {$\frac{\phi^n}{\sqrt{5}}$};
  \end{tikzpicture}
\end{center}

This allows to conclude that $\frac{\phi^n}{\sqrt{5}}$ is always
within rounding error of the actual Fibonacci number. Hence, by
rounding $\frac{\phi^n}{\sqrt{5}}$ we get the $n$-th Fibonacci
number.

This completes our derivation for \ref{fib_num}.

\vspace{1em}

We can prove \ref{fib_sum} by using an inductive argument.

\vspace{1em}

\textit{Argument.}

Base case ($n = 1$): 

$$LHS = \sum_{i = 1}^{1} F_i = F_1 = 1$$ 
$$RHS = F_{1 + 2} - 1 = F_3 - 1 = F_2 + F_1 - 1 = 1 + 1 - 1 =
1$$

Hence, the base case holds since $LHS = RHS$.

\vspace{1em}

Inductive case ($n = k$):

Suppose that sum holds for $n = k - 1$. 

\begin{align}
  \sum_{i = 1}^{k - 1} F_i = F_{k - 1 + 2} - 1 = F_{k + 1} - 1
  \label{ind_hyp}
\end{align}

We are required to show that the sum holds for $n = k$.

\begin{align*}
  \sum_{i = 1}^{k} F_i & = F_{k} + \sum_{i = 1}^{k - 1} F_i\\
                       % & \qquad (By\ \ref{ind_hyp})\\
                       & = F_{k + 1} + F_{k} - 1\\
                       % & \qquad (By\ \ref{fib_def})\\
                       & = F_{k + 2} - 1 
\end{align*}

Therefore, by induction, \ref{fib_sum} holds for all natural
numbers.

\chapter{Chapter}
This is a chapter
\section{Section g}
This is a section
\subsection{Sub Section}
This is a sub section
\subsubsection{Sub Sub Section}
This is a sub sub section

\begin{itemize}
  \item this is an item
  \item this is also an item
  \begin{itemize}
    \item Subitem A
    \item Subitem B
    \begin{itemize}
      \item Subsubitem A
      \item Subsubitem B
      \begin{itemize}
        \item Subsubsubitem A
        \item Subsubsubitem B
      \end{itemize}
    \end{itemize}
  \end{itemize}
\end{itemize}

\begin{lstlisting}[caption={The section in \texttt{emitString()}
which handles escaping.},language=C]
if (ch == '\\') {
  switch (ch = CSNext()) {
    case '\\':
    case '"':
      // DO NOTHING
      break;
    default:
      fprintf(
          stderr,
          "emitString: Invalid escape sequence\n");
      return -1;
  }
}

if (tok.len >= alc) {
  alc += GROW_SIZE;
  char *tmp = realloc(tok.str, sizeof(char) * alc);

  if (tmp == NULL) {
    perror("emitString");
    return -1;
  } else {
    tok.str = tmp;
  }
}
\end{lstlisting}

\begin{prop}
  hello
\end{prop}
\begin{thm}
  hello
\end{thm}
\begin{conj}
  hello
\end{conj}
\begin{cor}
  hello
\end{cor}
\begin{lma}
  hello
\end{lma}


\appendix

\chapter{Is this part of the appendix?}
\section{This is in the appendix}
\chapter{Is this also part of the appendix?}


\end{document}
